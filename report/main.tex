% -------------------------
%	Author: ToanHac
%	Created: 09.2024
% -------------------------
\documentclass[a4paper]{article}
\usepackage{a4wide,amssymb,epsfig,latexsym,multicol,array,hhline,fancyhdr}
\usepackage{vntex}
\usepackage{amsmath}
\usepackage{amssymb}
\usepackage{lastpage}
\usepackage{indentfirst}
\usepackage[lined,boxed,commentsnumbered]{algorithm2e}
\usepackage{enumerate}
\usepackage{color}
\usepackage{graphicx}							% Standard graphics package
\usepackage{array}
\usepackage{tabularx, caption}
\usepackage{multirow}
\usepackage{multicol}
\usepackage{rotating}
\usepackage{graphics}
\usepackage{geometry}
\usepackage{setspace}
\usepackage{epsfig}
\usepackage{tikz}
\usetikzlibrary{arrows,snakes,backgrounds}
\usepackage{hyperref}
\hypersetup{urlcolor=blue,linkcolor=black,citecolor=black,colorlinks=true} 
%\usepackage{pstcol} 								% PSTricks with the standard color package
\usepackage{subcaption}
\usepackage{pgfplots}
\pgfplotsset{compat=1.15}
\usepackage{mathrsfs}
\usetikzlibrary{arrows}
\pagestyle{empty}

\usepackage{listings}
% -- Defining colors:
\definecolor{codegray}{rgb}{0.5,0.5,0.5}
\definecolor{codegreen}{rgb}{0,0.6,0}
\definecolor{codepurple}{rgb}{0.58,0,0.82}
\definecolor{bg}{HTML}{F6F8FA} % Background color
\definecolor{keyword}{HTML}{D73A49} % Keyword color
\definecolor{identifier}{HTML}{005CC5} % Identifier color
\definecolor{builtin}{HTML}{6F42C1} % Builtin color
% Definig a custom style:
%For python code
\lstdefinestyle{python}{
    language=Python,
    backgroundcolor=\color{bg},
    commentstyle=\color{codegreen},
    keywordstyle=\color{keyword}\bfseries,
    stringstyle=\color{codepurple},
    numberstyle=\tiny\color{codegray},
    identifierstyle=\color{identifier},
    basicstyle=\ttfamily\small,
    breaklines=true,
    keepspaces=true,                 
    numbers=left,       
    numbersep=5pt,
    frame=single,
    showspaces=false,
    rulecolor=\color{bg},
    tabsize=2,
    showstringspaces=false,
    morekeywords={self},
    framexleftmargin=3.5mm,
    linewidth=0.6\linewidth,
    xleftmargin=12pt,
    aboveskip=12pt,
    belowskip=12pt,
    rulecolor=\color{codegray},
}

\DeclareMathOperator*{\argmax}{argmax}
\DeclareMathOperator*{\argmin}{argmin}

\AtBeginDocument{\renewcommand*\contentsname{Mục lục}}
\AtBeginDocument{\renewcommand*\refname{Tài liệu tham khảo}}
%\usepackage{fancyhdr}
\setlength{\headheight}{40pt}
\pagestyle{fancy}
\fancyhead{} % clear all header fields
\fancyhead[L]{
 \begin{tabular}{rl}
    \begin{picture}(25,15)(0,0)
    \put(0,-8){\includegraphics[width=8mm, height=8mm]{Images/hcmut.png}}
    %\put(0,-8){\epsfig{width=10mm,figure=hcmut.eps}}
   \end{picture}&
	%\includegraphics[width=8mm, height=8mm]{hcmut.png} & %
	\begin{tabular}{l}
		\textbf{\bf \ttfamily Trường Đại học Bách khoa, Tp. Hồ Chí Minh}\\
		\textbf{\bf \ttfamily Khoa Khoa học và Kỹ thuật máy tính}
	\end{tabular} 	
 \end{tabular}
}
\fancyhead[R]{
	\begin{tabular}{l}
		\tiny \bf \\
		\tiny \bf 
	\end{tabular}  }
\fancyfoot{} % clear all footer fields
\fancyfoot[L]{\scriptsize \ttfamily Bài tập lớn Công nghệ phần mềm - Học kỳ 241}
\fancyfoot[R]{\scriptsize \ttfamily Trang {\thepage}/\pageref{LastPage}}
\renewcommand{\headrulewidth}{0.3pt}
\renewcommand{\footrulewidth}{0.3pt}


%%%
\setcounter{secnumdepth}{4}
\setcounter{tocdepth}{3}
\makeatletter
\newcounter {subsubsubsection}[subsubsection]
\renewcommand\thesubsubsubsection{\thesubsubsection .\@alph\c@subsubsubsection}
\newcommand\subsubsubsection{\@startsection{subsubsubsection}{4}{\z@}%
                                     {-3.25ex\@plus -1ex \@minus -.2ex}%
                                     {1.5ex \@plus .2ex}%
                                     {\normalfont\normalsize\bfseries}}
\newcommand*\l@subsubsubsection{\@dottedtocline{3}{10.0em}{4.1em}}
\newcommand*{\subsubsubsectionmark}[1]{}
\makeatother


\begin{document}

\begin{titlepage}
\begin{tikzpicture}[remember picture,overlay,inner sep=0,outer sep=0]
	\draw[blue!70!black,line width=4pt] ([xshift=-1.5cm,yshift=-2cm]current page.north east) coordinate (A)--([xshift=1.5cm,yshift=-2cm]current page.north west) coordinate(B)--([xshift=1.5cm,yshift=2cm]current page.south west) coordinate (C)--([xshift=-1.5cm,yshift=2cm]current page.south east) coordinate(D)--cycle;
\end{tikzpicture}
\begin{center}
\textbf{\large ĐẠI HỌC QUỐC GIA TP. HỒ CHÍ MINH} \\
\textbf{\large TRƯỜNG ĐẠI HỌC BÁCH KHOA} \\
\textbf{\large KHOA KHOA HỌC VÀ KỸ THUẬT MÁY TÍNH}
\end{center}


\vspace{1cm}

\begin{figure}[h!]
\begin{center}
\includegraphics[width=3cm]{Images/hcmut.png}
\end{center}
\end{figure}

\vspace{1cm}


\begin{center}
\begin{tabular}{c}
\multicolumn{1}{l}{\textbf{{\Large Công nghệ phần mềm}}}\\
~~\\
\hline
\\
\multicolumn{1}{l}{\textbf{{\Large Bài tập lớn} (Nhóm x - L02)}}\\
\\
\textbf{{\Huge Student Smart Printing Service}}\\
\\
\textbf{{\Huge HCMUT\_SSPS}}\\
\\
\hline
\end{tabular}
\end{center}

\vspace{0.3cm}

	\begin{table}[h]
		\centering
		\begin{tabular}{rrlcl}
			\hspace{2.25 cm} & GVHD: & Trương Trần Tuấn Phát & & \\
			& SV thực hiện: & ABC & -- & xxx \\
            & & ABC & -- & xxx \\
		\end{tabular}
	\end{table}


\begin{center}
{\footnotesize TP. HỒ CHÍ MINH, THÁNG x, 20xx}
\end{center}
\end{titlepage}


%\thispagestyle{empty}
%%%%%%%%%%%%%%%%%%%%%%%
\newpage
\tableofcontents
\newpage
% === Begin random content === %
\section*{Lời mở đầu}

% === End random content === %
% === Being main content === %
\newpage
\section{Task 1.1}
\subsection{Bối cảnh}
Hiện nay, in ấn tài liệu vẫn là một nhu cầu thiết yếu trong môi trường giáo dục nói chung và Trường Đại học Bách khoa - Đại học Quốc gia TP.HCM nói riêng, khi sinh viên thường xuyên cần các tài liệu in ấn phục vụ cho việc học tập và thi cử. Nhu cầu này đặc biệt gia tăng vào các giờ cao điểm, như mùa thi hoặc đầu học kỳ, khi sinh viên cần in slide bài giảng, tài liệu học tập, hoặc tài liệu được mang vào phòng thi. Tuy nhiên, hiện tại sinh viên có thể in ấn thông qua 2 cách là in tại trường (ví dụ như thư viện BK.B1 ở cs2) hoặc là ra các tiệm in bên ngoài. Đối với lựa chọn thứ nhất là in tại trường sinh viên thường gặp phải tình trạng quá tải khiến cho việc in ấn mất rất nhiều thời gian đặc việc vào những lúc cao điểm, để tránh việc quá tải gây mất thời gian sinh viên có thể lựa chọn in ấn ở các tiệm bên ngoài. Tuy nhiên phương án này tiềm ẩn các nguy cơ rò rỉ các thông tin cá nhân (bài tập lớn, đồ án, ...) và gặp phải vấn đề bản quyền đặc biệt các tài liệu, slide của giảng viên trường đều có bản quyền của riêng họ. Vì vậy, việc triển khai một hệ thống in ấn thông minh, tích hợp ngay tại trường sẽ giúp đáp ứng kịp thời nhu cầu của sinh viên, đồng thời bảo đảm tính bảo mật và thuận tiện hơn trong việc in ấn tài liệu.

\subsection{Các bên liên quan và nhu cầu}
Các bên liên quan: sinh viên, nhân viên quản lý hệ thống máy in (SPSS), nhà trường.
Nhu cầu:
\begin{itemize}
    \item[$-$] Sinh viên:
    \begin{itemize}
        \item[$+$] In ấn tài liệu nhanh chóng.
        \item[$+$] Bảo mật các tài liệu mà sinh viên upload lên để in.
        \item[$+$] Quản lý thông tin in ấn (lịch sử, quản lý, số trang giấy, ...).
    \end{itemize}
    \item[$-$] Nhân viên quản lý hệ thống máy in:
    \begin{itemize}
        \item[$+$] Chịu trách nhiệm cấu hình hệ thống.
        \item[$+$] Quản lý máy in (tắt mở, xem trạng thái máy in, ...).
        \item[$+$] Xử lý yêu cầu sinh viên.
    \end{itemize}
    \item[$-$] Nhà trường:
    \begin{itemize}
        \item[$+$] Tối ưu hóa việc sử dụng cơ sở hạ tầng in ấn.
        \item[$+$] Lập các báo cáo dịnh kỳ để theo dõi và quản lý.
        \item[$+$] Hỗ trợ và quản lý chung toàn hệ thống.
    \end{itemize}
\end{itemize}
Lợi ích:
\begin{itemize}
    \item[$-$] Sinh viên:
    \begin{itemize}
        \item[$+$] In ấn tài liệu thuật tiện và nhanh chóng.
        \item[$+$] Bảo mật các tài liệu và thông tin cá nhân.
        \item[$+$] Giá thành hợp lý, thanh toán thuận tiện.
    \end{itemize}
    \item[$-$] Nhân viên quản lý hệ thống máy in:
    \begin{itemize}
        \item[$+$] Tự động hoá quy trình in ấn giảm các công việc cần làm.
        \item[$+$] Quản lý máy in (tắt mở, xem trạng thái máy in, ...).
    \end{itemize}
    \item[$-$] Nhà trường:
    \begin{itemize}
        \item[$+$] Đảm bảo tính bảo mật của các tài liệu nội bộ.
        \item[$+$] Quản lý và tối ưu hoá việc sử dụng máy in một cách hiệu quả
    \end{itemize}
\end{itemize}
% === End main content === %
\newpage
\end{document}

