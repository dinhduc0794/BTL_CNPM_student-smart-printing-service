% -------------------------
%	Author: ToanHac
%	Created: 09.2024
% -------------------------
\documentclass[a4paper]{article}
\usepackage{a4wide,amssymb,epsfig,latexsym,multicol,array,hhline,fancyhdr}
\usepackage{vntex}
\usepackage{amsmath}
\usepackage{amssymb}
\usepackage{lastpage}
\usepackage{indentfirst}
\usepackage[lined,boxed,commentsnumbered]{algorithm2e}
\usepackage{enumerate}
\usepackage{color}
\usepackage{graphicx}							% Standard graphics package
\usepackage{array}
\usepackage{tabularx, caption}
\usepackage{multirow}
\usepackage{multicol}
\usepackage{rotating}
\usepackage{graphics}
\usepackage{geometry}
\usepackage{setspace}
\usepackage{epsfig}
\usepackage{tikz}
\usetikzlibrary{arrows,snakes,backgrounds}
\usepackage{hyperref}
\hypersetup{urlcolor=blue,linkcolor=black,citecolor=black,colorlinks=true} 
%\usepackage{pstcol} 								% PSTricks with the standard color package
\usepackage{subcaption}
\usepackage{pgfplots}
\pgfplotsset{compat=1.15}
\usepackage{mathrsfs}
\usetikzlibrary{arrows}
\pagestyle{empty}

\usepackage{listings}
% -- Defining colors:
\definecolor{codegray}{rgb}{0.5,0.5,0.5}
\definecolor{codegreen}{rgb}{0,0.6,0}
\definecolor{codepurple}{rgb}{0.58,0,0.82}
\definecolor{bg}{HTML}{F6F8FA} % Background color
\definecolor{keyword}{HTML}{D73A49} % Keyword color
\definecolor{identifier}{HTML}{005CC5} % Identifier color
\definecolor{builtin}{HTML}{6F42C1} % Builtin color
% Definig a custom style:
%For python code
\lstdefinestyle{python}{
    language=Python,
    backgroundcolor=\color{bg},
    commentstyle=\color{codegreen},
    keywordstyle=\color{keyword}\bfseries,
    stringstyle=\color{codepurple},
    numberstyle=\tiny\color{codegray},
    identifierstyle=\color{identifier},
    basicstyle=\ttfamily\small,
    breaklines=true,
    keepspaces=true,                 
    numbers=left,       
    numbersep=5pt,
    frame=single,
    showspaces=false,
    rulecolor=\color{bg},
    tabsize=2,
    showstringspaces=false,
    morekeywords={self},
    framexleftmargin=3.5mm,
    linewidth=0.6\linewidth,
    xleftmargin=12pt,
    aboveskip=12pt,
    belowskip=12pt,
    rulecolor=\color{codegray},
}

\DeclareMathOperator*{\argmax}{argmax}
\DeclareMathOperator*{\argmin}{argmin}

\AtBeginDocument{\renewcommand*\contentsname{Mục lục}}
\AtBeginDocument{\renewcommand*\refname{Tài liệu tham khảo}}
%\usepackage{fancyhdr}
\setlength{\headheight}{40pt}
\pagestyle{fancy}
\fancyhead{} % clear all header fields
\fancyhead[L]{
 \begin{tabular}{rl}
    \begin{picture}(25,15)(0,0)
    \put(0,-8){\includegraphics[width=8mm, height=8mm]{Images/hcmut.png}}
    %\put(0,-8){\epsfig{width=10mm,figure=hcmut.eps}}
   \end{picture}&
	%\includegraphics[width=8mm, height=8mm]{hcmut.png} & %
	\begin{tabular}{l}
		\textbf{\bf \ttfamily Trường Đại học Bách khoa, Tp. Hồ Chí Minh}\\
		\textbf{\bf \ttfamily Khoa Khoa học và Kỹ thuật máy tính}
	\end{tabular} 	
 \end{tabular}
}
\fancyhead[R]{
	\begin{tabular}{l}
		\tiny \bf \\
		\tiny \bf 
	\end{tabular}  }
\fancyfoot{} % clear all footer fields
\fancyfoot[L]{\scriptsize \ttfamily Bài tập lớn Công nghệ phần mềm - Học kỳ 241}
\fancyfoot[R]{\scriptsize \ttfamily Trang {\thepage}/\pageref{LastPage}}
\renewcommand{\headrulewidth}{0.3pt}
\renewcommand{\footrulewidth}{0.3pt}


%%%
\setcounter{secnumdepth}{4}
\setcounter{tocdepth}{3}
\makeatletter
\newcounter {subsubsubsection}[subsubsection]
\renewcommand\thesubsubsubsection{\thesubsubsection .\@alph\c@subsubsubsection}
\newcommand\subsubsubsection{\@startsection{subsubsubsection}{4}{\z@}%
                                     {-3.25ex\@plus -1ex \@minus -.2ex}%
                                     {1.5ex \@plus .2ex}%
                                     {\normalfont\normalsize\bfseries}}
\newcommand*\l@subsubsubsection{\@dottedtocline{3}{10.0em}{4.1em}}
\newcommand*{\subsubsubsectionmark}[1]{}
\makeatother


\begin{document}

\begin{titlepage}
\begin{tikzpicture}[remember picture,overlay,inner sep=0,outer sep=0]
	\draw[blue!70!black,line width=4pt] ([xshift=-1.5cm,yshift=-2cm]current page.north east) coordinate (A)--([xshift=1.5cm,yshift=-2cm]current page.north west) coordinate(B)--([xshift=1.5cm,yshift=2cm]current page.south west) coordinate (C)--([xshift=-1.5cm,yshift=2cm]current page.south east) coordinate(D)--cycle;
\end{tikzpicture}
\begin{center}
\textbf{\large ĐẠI HỌC QUỐC GIA TP. HỒ CHÍ MINH} \\
\textbf{\large TRƯỜNG ĐẠI HỌC BÁCH KHOA} \\
\textbf{\large KHOA KHOA HỌC VÀ KỸ THUẬT MÁY TÍNH}
\end{center}


\vspace{1cm}

\begin{figure}[h!]
\begin{center}
\includegraphics[width=3cm]{Images/hcmut.png}
\end{center}
\end{figure}

\vspace{1cm}


\begin{center}
\begin{tabular}{c}
\multicolumn{1}{l}{\textbf{{\Large Công nghệ phần mềm}}}\\
~~\\
\hline
\\
\multicolumn{1}{l}{\textbf{{\Large Bài tập lớn} (Nhóm x - L02)}}\\
\\
\textbf{{\Huge Student Smart Printing Service}}\\
\\
\textbf{{\Huge HCMUT\_SSPS}}\\
\\
\hline
\end{tabular}
\end{center}

\vspace{0.3cm}

	\begin{table}[h]
		\centering
		\begin{tabular}{rrlcl}
			\hspace{2.25 cm} & GVHD: & Trương Trần Tuấn Phát & & \\
			& SV thực hiện: & Hoa Toàn Hạc & -- & 2201917 \\
		\end{tabular}
	\end{table}


\begin{center}
{\footnotesize TP. HỒ CHÍ MINH, THÁNG x, 20xx}
\end{center}
\end{titlepage}


%\thispagestyle{empty}
%%%%%%%%%%%%%%%%%%%%%%%
\newpage
\tableofcontents
\newpage
% === Begin random content === %
\section*{Lời mở đầu}
aaaa
% === End random content === %
% === Being main content === %
\newpage

% === End main content === %
\newpage
\begin{thebibliography}{80}
\bibitem{bib1}
Vũ Hữu Tiệp, "Machine Learning cơ bản," GitHub, Jan. 2, 2022. [Online]. Available: https://github.com/tiepvupsu/ebookMLCB.
\end{thebibliography}
\end{document}

