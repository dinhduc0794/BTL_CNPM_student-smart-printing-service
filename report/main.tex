% -------------------------
%	Author: ToanHac
%	Created: 09.2024
% -------------------------
\documentclass[a4paper]{article}
\usepackage{a4wide,amssymb,epsfig,latexsym,multicol,array,hhline,fancyhdr}
\usepackage{vntex}
\usepackage{amsmath}
\usepackage{amssymb}
\usepackage{lastpage}
\usepackage{indentfirst}
\usepackage[lined,boxed,commentsnumbered]{algorithm2e}
\usepackage{enumerate}
\usepackage{color}
\usepackage{graphicx}							% Standard graphics package
\usepackage{array}
\usepackage{tabularx, caption}
\usepackage{multirow}
\usepackage{multicol}
\usepackage{rotating}
\usepackage{graphics}
\usepackage{geometry}
\usepackage{setspace}
\usepackage{epsfig}
\usepackage{tikz}
\usetikzlibrary{arrows,snakes,backgrounds}
\usepackage{hyperref}
\hypersetup{urlcolor=blue,linkcolor=black,citecolor=black,colorlinks=true} 
%\usepackage{pstcol} 								% PSTricks with the standard color package
\usepackage{subcaption}
\usepackage{pgfplots}
\pgfplotsset{compat=1.15}
\usepackage{mathrsfs}
\usetikzlibrary{arrows}
\pagestyle{empty}

\usepackage{listings}
% -- Defining colors:
\definecolor{codegray}{rgb}{0.5,0.5,0.5}
\definecolor{codegreen}{rgb}{0,0.6,0}
\definecolor{codepurple}{rgb}{0.58,0,0.82}
\definecolor{bg}{HTML}{F6F8FA} % Background color
\definecolor{keyword}{HTML}{D73A49} % Keyword color
\definecolor{identifier}{HTML}{005CC5} % Identifier color
\definecolor{builtin}{HTML}{6F42C1} % Builtin color
% Definig a custom style:
%For python code
\lstdefinestyle{python}{
    language=Python,
    backgroundcolor=\color{bg},
    commentstyle=\color{codegreen},
    keywordstyle=\color{keyword}\bfseries,
    stringstyle=\color{codepurple},
    numberstyle=\tiny\color{codegray},
    identifierstyle=\color{identifier},
    basicstyle=\ttfamily\small,
    breaklines=true,
    keepspaces=true,                 
    numbers=left,       
    numbersep=5pt,
    frame=single,
    showspaces=false,
    rulecolor=\color{bg},
    tabsize=2,
    showstringspaces=false,
    morekeywords={self},
    framexleftmargin=3.5mm,
    linewidth=0.6\linewidth,
    xleftmargin=12pt,
    aboveskip=12pt,
    belowskip=12pt,
    rulecolor=\color{codegray},
}

\DeclareMathOperator*{\argmax}{argmax}
\DeclareMathOperator*{\argmin}{argmin}

\AtBeginDocument{\renewcommand*\contentsname{Mục lục}}
\AtBeginDocument{\renewcommand*\refname{Tài liệu tham khảo}}
%\usepackage{fancyhdr}
\setlength{\headheight}{40pt}
\pagestyle{fancy}
\fancyhead{} % clear all header fields
\fancyhead[L]{
 \begin{tabular}{rl}
    \begin{picture}(25,15)(0,0)
    \put(0,-8){\includegraphics[width=8mm, height=8mm]{Images/hcmut.png}}
    %\put(0,-8){\epsfig{width=10mm,figure=hcmut.eps}}
   \end{picture}&
	%\includegraphics[width=8mm, height=8mm]{hcmut.png} & %
	\begin{tabular}{l}
		\textbf{\bf \ttfamily Trường Đại học Bách khoa, Tp. Hồ Chí Minh}\\
		\textbf{\bf \ttfamily Khoa Khoa học và Kỹ thuật máy tính}
	\end{tabular} 	
 \end{tabular}
}
\fancyhead[R]{
	\begin{tabular}{l}
		\tiny \bf \\
		\tiny \bf 
	\end{tabular}  }
\fancyfoot{} % clear all footer fields
\fancyfoot[L]{\scriptsize \ttfamily Bài tập lớn Công nghệ phần mềm - Học kỳ 241}
\fancyfoot[R]{\scriptsize \ttfamily Trang {\thepage}/\pageref{LastPage}}
\renewcommand{\headrulewidth}{0.3pt}
\renewcommand{\footrulewidth}{0.3pt}


%%%
\setcounter{secnumdepth}{4}
\setcounter{tocdepth}{3}
\makeatletter
\newcounter {subsubsubsection}[subsubsection]
\renewcommand\thesubsubsubsection{\thesubsubsection .\@alph\c@subsubsubsection}
\newcommand\subsubsubsection{\@startsection{subsubsubsection}{4}{\z@}%
                                     {-3.25ex\@plus -1ex \@minus -.2ex}%
                                     {1.5ex \@plus .2ex}%
                                     {\normalfont\normalsize\bfseries}}
\newcommand*\l@subsubsubsection{\@dottedtocline{3}{10.0em}{4.1em}}
\newcommand*{\subsubsubsectionmark}[1]{}
\makeatother


\begin{document}

\begin{titlepage}
\begin{tikzpicture}[remember picture,overlay,inner sep=0,outer sep=0]
	\draw[blue!70!black,line width=4pt] ([xshift=-1.5cm,yshift=-2cm]current page.north east) coordinate (A)--([xshift=1.5cm,yshift=-2cm]current page.north west) coordinate(B)--([xshift=1.5cm,yshift=2cm]current page.south west) coordinate (C)--([xshift=-1.5cm,yshift=2cm]current page.south east) coordinate(D)--cycle;
\end{tikzpicture}
\begin{center}
\textbf{\large ĐẠI HỌC QUỐC GIA TP. HỒ CHÍ MINH} \\
\textbf{\large TRƯỜNG ĐẠI HỌC BÁCH KHOA} \\
\textbf{\large KHOA KHOA HỌC VÀ KỸ THUẬT MÁY TÍNH}
\end{center}


\vspace{1cm}

\begin{figure}[h!]
\begin{center}
\includegraphics[width=3cm]{Images/hcmut.png}
\end{center}
\end{figure}

\vspace{1cm}


\begin{center}
\begin{tabular}{c}
\multicolumn{1}{l}{\textbf{{\Large Công nghệ phần mềm}}}\\
~~\\
\hline
\\
\multicolumn{1}{l}{\textbf{{\Large Bài tập lớn} (Nhóm x - L02)}}\\
\\
\textbf{{\Huge Student Smart Printing Service}}\\
\\
\textbf{{\Huge HCMUT SSPS}}\\
\\
\hline
\end{tabular}
\end{center}

\vspace{0.3cm}

	\begin{table}[h]
		\centering
		\begin{tabular}{rrlcl}
			\hspace{2.25 cm} & GVHD: & Trương Trần Tuấn Phát & & \\
			& SV thực hiện: & ABC & -- & xxx \\
            & & ABC & -- & xxx \\
		\end{tabular}
	\end{table}


\begin{center}
{\footnotesize TP. HỒ CHÍ MINH, THÁNG x, 20xx}
\end{center}
\end{titlepage}


%\thispagestyle{empty}
%%%%%%%%%%%%%%%%%%%%%%%
\newpage
\tableofcontents
\newpage
% === Begin random content === %
\section*{Lời mở đầu}

% === End random content === %
% === Being main content === %
\newpage
\section{Bối cảnh và các bên liên quan}
\subsection{Bối cảnh}
Hiện nay, in ấn tài liệu vẫn là một nhu cầu thiết yếu trong môi trường giáo dục nói chung và Trường Đại học Bách khoa - Đại học Quốc gia TP.HCM nói riêng, khi sinh viên thường xuyên cần các tài liệu in ấn phục vụ cho việc học tập và thi cử. Nhu cầu này đặc biệt gia tăng vào các giờ cao điểm, như mùa thi hoặc đầu học kỳ, khi sinh viên cần in slide bài giảng, tài liệu học tập, hoặc tài liệu được mang vào phòng thi. Tuy nhiên, hiện tại sinh viên có thể in ấn thông qua 2 cách là in tại trường (ví dụ như thư viện BK.B1 ở cs2) hoặc là ra các tiệm in bên ngoài. Đối với lựa chọn thứ nhất là in tại trường sinh viên thường gặp phải tình trạng quá tải khiến cho việc in ấn mất rất nhiều thời gian đặc việc vào những lúc cao điểm, để tránh việc quá tải gây mất thời gian sinh viên có thể lựa chọn in ấn ở các tiệm bên ngoài. Tuy nhiên phương án này tiềm ẩn các nguy cơ rò rỉ các thông tin cá nhân (bài tập lớn, đồ án, ...) và gặp phải vấn đề bản quyền đặc biệt các tài liệu, slide của giảng viên trường đều có bản quyền của riêng họ. Vì vậy, việc triển khai một hệ thống in ấn thông minh SSPS, tích hợp ngay tại trường sẽ giúp đáp ứng kịp thời nhu cầu của sinh viên, đồng thời bảo đảm tính bảo mật và thuận tiện hơn trong việc in ấn tài liệu.

\subsection{Các bên liên quan và nhu cầu}
\paragraph*{Các bên liên quan:}
\begin{itemize}
    \item[$-$] Sinh viên: sinh viên đang học tập tại trường.
    \item[$-$] Nhà trường: Trường Đại học Bách khoa - Đại học Quốc gia TP.HCM.
    \item[$-$] Nhân viên quản lý hệ thống máy in: người quản lý, hỗ trợ hệ thống vận hành.
\end{itemize}
\paragraph*{Nhu cầu}
\begin{itemize}
    \item[$-$] Sinh viên cần một hệ thống in ấn đáp ứng các nhu cầu sau:
    \begin{itemize}
        \item[$+$] \textbf{In ấn nhanh chóng:} Yêu cầu hệ thống giúp sinh viên in tài liệu một cách nhanh gọn, không mất thời gian chờ đợi, đặc biệt là khi cần tài liệu gấp cho các buổi học, thuyết trình, hay nộp bài.
        \item[$+$] \textbf{Bảo mật tài liệu:} Đảm bảo tính riêng tư và an toàn cho các tài liệu mà sinh viên tải lên để in. Chỉ người dùng có thẩm quyền mới có thể truy cập và in tài liệu, giúp tránh rò rỉ thông tin quan trọng.
        \item[$+$] \textbf{Quản lý thông tin in ấn:} Hệ thống cần cung cấp chức năng theo dõi lịch sử in ấn, quản lý số trang giấy đã sử dụng, và có thể giám sát quá trình in để giúp sinh viên kiểm soát chi phí cũng như hiệu suất sử dụng tài liệu của mình.
    \end{itemize}
    \item[$-$] Nhà trường:
    \begin{itemize}
        \item[$+$] \textbf{Tối ưu hóa việc sử dụng cơ sở hạ tầng in ấn:} Đảm bảo hệ thống in ấn hoạt động hiệu quả, tối ưu hóa việc sử dụng thiết bị và tài nguyên như giấy, mực in, từ đó giảm thiểu lãng phí và chi phí vận hành.
        \item[$+$] \textbf{Theo dõi và quản lý việc in ấn:} Cần có hệ thống theo dõi chi tiết lượng in ấn, quản lý người dùng, và phân bổ hợp lý tài nguyên in ấn dựa trên nhu cầu sử dụng. Điều này giúp nhà trường dễ dàng quản lý chi phí và cải thiện hiệu suất in ấn tổng thể.
        \item[$+$] \textbf{Đảm bảo vấn đề bản quyền của các tài liệu nội bộ:} Hệ thống cần kiểm soát chặt chẽ việc sao chép, phân phối các tài liệu nội bộ, nhằm đảm bảo việc tuân thủ bản quyền và giữ gìn sự bảo mật cho các tài liệu thuộc sở hữu của nhà trường.
    \end{itemize}
\end{itemize}
\paragraph*{Lợi ích}
\begin{itemize}
    \item[$-$] Sinh viên:
    \begin{itemize}
        \item[$+$] \textbf{In ấn tài liệu thuận tiện và nhanh chóng:} Sinh viên có thể dễ dàng tải tài liệu và in ấn mà không cần phải chờ đợi lâu, hỗ trợ tốt cho việc học tập và nộp bài đúng hạn.
        \item[$+$] \textbf{Bảo mật các tài liệu và thông tin cá nhân:} Hệ thống đảm bảo quyền riêng tư và bảo mật đối với các tài liệu quan trọng, giúp sinh viên an tâm khi sử dụng.
        \item[$+$] \textbf{Giá thành hợp lý, thanh toán thuận tiện:} Hệ thống cung cấp mức giá hợp lý và hỗ trợ thanh toán dễ dàng, phù hợp với nhu cầu của sinh viên.
    \end{itemize}
    \item[$-$] Nhân viên quản lý hệ thống máy in:
    \begin{itemize}
        \item[$+$] \textbf{Tự động hoá quy trình in ấn, giảm bớt công việc cần làm:} Nhân viên quản lý có thể giảm thiểu công sức và thời gian dành cho việc điều hành hệ thống in ấn nhờ quy trình tự động hóa.
        \item[$+$] \textbf{Quản lý máy in hiệu quả:} Hệ thống cung cấp công cụ để nhân viên dễ dàng theo dõi trạng thái máy in, tắt mở thiết bị từ xa, và xử lý các sự cố nhanh chóng.
    \end{itemize}
    \item[$-$] Nhà trường:
    \begin{itemize}
        \item[$+$] \textbf{Đảm bảo tính bảo mật của các tài liệu nội bộ:} Nhà trường có thể kiểm soát chặt chẽ việc in ấn và sao chép tài liệu, đảm bảo các tài liệu nội bộ được bảo mật tối đa.
        \item[$+$] \textbf{Quản lý và tối ưu hoá việc sử dụng máy in một cách hiệu quả:} Hệ thống giúp nhà trường quản lý chi phí, tài nguyên và sử dụng cơ sở hạ tầng in ấn một cách thông minh và hiệu quả, giảm lãng phí và tăng cường hiệu suất.
    \end{itemize}
\end{itemize}
\section{Yêu cầu với hệ thống}
\subsection{Yêu cầu chức năng}
\paragraph*{Đối với Sinh viên}
\begin{itemize}
    \item Đăng nhập thông qua hệ thống SSO của trường.
    \item Có thể tải lên (upload) tài liệu trong những định dạng cho phép để in.
    \item Có thể chọn máy in và thiết lập tuỳ chọn in (khổ giấy, 2 mặt hay 1 mặt, số lượng bản sao, ...).
    \item Có thể xem lịch sử in ấn của bản thân cũng như báo cáo cơ bản về việc sử dụng máy in của bản thân.
    \item Có thể hẹn lịch tới lấy tài liệu khi đã gửi lệnh in.
    \item Thanh toán thông qua BKPay của trường.
\end{itemize}

\paragraph*{Đối với Nhân viên quản lý hệ thống máy in}
\begin{itemize}
    \item Đăng nhập thông qua hệ thống SSO của trường.
    \item Có thể quản lý máy in (thêm, kích hoạt, tạm dừng, ...).
    \item Có thể xem được lịch sử in ấn của sinh viên.
    \item Có thể quản lý sinh viên sử dụng dịch vụ (cấm dùng, tạm thời không cho phép dùng, ...).
    \item Quản lý số trang in của mỗi sinh viên (cấp phát, thu hồi, ...).
\end{itemize}
\subsection{Yêu cầu phi chức năng}
\begin{itemize}
    \item Giao diện dễ dàng sử dụng, các nút chức năng rõ ràng tránh gây hiểu lầm.
    \item Tương thích nhiều loại thiết bị (responsive).
    \item Thời gian phản hồi của hệ thống phải nhanh (upload tài liệu, thanh toán, ...)
    \item Ở các chức năng quan trọng như upload tài liệu, ra lệnh in cần có thông báo thành công khi hoàn thành.
    \item Tính bảo mật cao (bắt buộc đăng nhập thông qua SSO).
    \item Hệ thống phải có khả năng mở rộng dễ dàng (tăng thêm số lượng máy in, ...).
    \item Độ tin cậy cao đảm bảo phải ghi nhận lại các lỗi xảy ra (nếu có) trong quá trình sử dụng của sinh viên và thông báo tới sinh viên.
    \item Hệ thống phải khả dụng tránh quá tải, sập thời gian dài kể cả những lúc cao điểm như trước kỳ thi.
    \item Hệ thống phải dễ dàng bảo trì, nâng cấp (các bản vá phải được cập mà không làm gián đoạn hệ thống quá 1 giờ).
    \item Phải có hướng dẫn sử dụng chi tiết bằng tiếng Anh và tiếng Việt.
    \item Phải thích hợp mượt mà với hệ thống thanh toán BKPay.
\end{itemize}
% === End main content === %
\newpage
\end{document}

